\documentclass[a4paper, 11pt]{article}
\usepackage[margin=3cm]{geometry}
\usepackage[]{fontenc}
\usepackage[utf8]{inputenc}
\usepackage[italian]{babel}
\usepackage{geometry}
\geometry{a4paper, top=2cm, bottom=3cm, left=1.5cm, right=1.5cm, heightrounded, bindingoffset=5mm}
\usepackage{amsmath}
\usepackage{amssymb}
\usepackage{gensymb}
\usepackage{graphicx}
\usepackage{psfrag,amsmath,amsfonts,verbatim}
\usepackage{xcolor}
\usepackage{color,soul}
\usepackage{fancyhdr}
\usepackage{indentfirst}
\usepackage{graphicx}
\usepackage{newlfont}
\usepackage{amssymb}
\usepackage{amsmath}
\usepackage{latexsym}
\usepackage{amsthm}
%\usepackage{subfigure}
\usepackage{subcaption}
\usepackage{psfrag}
\usepackage{footnote}
\usepackage{graphics}
\usepackage{color}
\usepackage{hyperref}
\usepackage{tikz}
\usepackage{float}


\usetikzlibrary{snakes}
\usetikzlibrary{positioning}
\usetikzlibrary{shapes,arrows}

	
	\tikzstyle{block} = [draw, fill=white, rectangle, 
	minimum height=3em, minimum width=6em]
	\tikzstyle{sum} = [draw, fill=white, circle, node distance=1cm]
	\tikzstyle{input} = [coordinate]
	\tikzstyle{output} = [coordinate]
	\tikzstyle{pinstyle} = [pin edge={to-,thin,black}]

\newcommand{\courseacronym}{CAT}
\newcommand{\coursename}{Controlli Automatici - T}
\newcommand{\tipology}{A }
\newcommand{\trace}{3}
\newcommand{\projectname}{Controllo trattamento farmacologico su popolazioni cellulari}
\newcommand{\group}{21}

%opening
\title{ \vspace{-1in}
		\huge \strut \coursename \strut 
		\\
		\Large  \strut Progetto Tipologia \tipology  - Traccia \trace 
		\\
		\Large  \strut \projectname\strut
		\\
		\Large  \strut Gruppo \group\strut
		\vspace{-0.4cm}
}
\author{Barone Leonardo, Del Giudice Domenico, Galli Francesco, Guzzonato Leonardo}
\date{Gennaio 2025}

\begin{document}



\maketitle
\vspace{-0.5cm}

Il progetto riguarda l’utilizzo di tecniche di controlli automatici per il trattamento farmacologico di cellule in ambiente di laboratorio.\newline \newline
\textbf{Descrizione del problema} \newline \newline
Si consideri un gruppo di cellule in cui sono presenti una densità di cellule $n_s(t)$ suscettibili al trattamento farmacologico e una densità di cellule $n_r(t)$ resistenti. Si supponga che la loro evoluzione sia descritta dalle
seguenti equazioni differenziali

%
\begin{subequations}\label{eq:system}
\begin{align}
	\dot{n}_s=r_s \bigl(1-\frac{n_s+n_r }{K}\bigr)\, n_s\, -m_s\, c_f\, n_s\, -\beta\, n_s\, +\gamma\, n_r\, -\alpha\, c_f\, n_s\ ,
\end{align}
\begin{align}
	\dot{n}_r=r_r \bigl(1-\frac{n_s+n_r }{K}\bigr)\, n_r\, -m_r\, c_f\, n_r\, +\beta\, n_s\, -\gamma\, n_r\, +\alpha\, c_f\, n_s\ ,
\end{align}
\end{subequations}
%


dove i parametri $r_s, r_r \in \mathbb{R}$ rappresentano i tassi di riproduzione delle due tipologie, mentre il parametro $K \in \mathbb{R}$ rappresenta la densità massima di cellule che l’ambiente può contenere. La variabile d’ingresso $c_f(t)$ indica la concentrazione del farmaco. In particolare, i parametri $m_s, m_r \in \mathbb{R}$ determinano, rispettivamente, la mortalità delle cellule suscettibili e quella delle cellule resistenti, con $m_s > m_r$. Tipicamente, le cellule possono mutare da una tipologia all’altra. Ad esempio, le cellule suscettibili possono diventare resistenti, come tenuto in conto dai termini $-\beta n_s$ nella prima equazione e $\beta n_s$ nella seconda equazione, con $\beta \in \mathbb{R}$. Analogamente, accade per le cellule resistenti attraverso il termine $\gamma n_r$, con $\gamma \in \mathbb{R}$. Infine, il termine $\alpha c_f n_r$ tiene conto delle cellule suscettibili che mutano in resistenti a seguito del trattamento farmacologico. Uno schema esplicativo è riportato in Figura~\eqref{eq:system}. 

\begin{figure}[h!]
    \centering
    \includegraphics[width=0.7\linewidth]{Figura1.png}
    \caption{Schema del modello~\eqref{eq:system} in cui sono rappresentati i flussi delle cellule}
    \label{fig:enter-label}
\end{figure}

Si supponga di poter misurare in ogni istante la densità di cellule resistenti $n_r(t)$


\clearpage
\section{Rappresentazione in Forma di Stato e Linearizzazione del Sistema intorno ad una coppia di equilibrio}

Si riporta il sistema~\eqref{eq:system} nella seguente forma di stato
%
\begin{subequations}
\begin{align}\label{eq:state_form}
	\dot{x} &= f(x,u)
	\\
	y &= h(x,u).
\end{align}
\end{subequations}
%
Pertanto, si individua lo stato $x$, l'ingresso $u$ e l'uscita $y$ del sistema come segue 
%
\begin{align*}
	x := \begin{bmatrix}
		n_s
		\\
		n_r
	\end{bmatrix}, \quad u := \begin{bmatrix}c_f\end{bmatrix}, \quad y := \begin{bmatrix}n_r\end{bmatrix}.
\end{align*}
%
Coerentemente con questa scelta, si ricava dal sistema~\eqref{eq:system} la seguente espressione per le funzioni $f$ ed $h$
%
\begin{align*}
	f(x,u) &:=\begin{bmatrix}
		 r_s \bigl(1-\frac{x_1+x_2}{K}\bigr)\, x_1\, -m_s\, u\, x_1\, -\beta\, x_1\, +\gamma\, x_2\, -\alpha\, u\, x_1\ 
	\\
	r_r \bigl(1-\frac{x_1+x_2 }{K}\bigr)\, x_2\, -m_r\, u\, x_2\, +\beta\, x_1\, -\gamma\, x_2\, +\alpha\, u\, x_1\ 
	\end{bmatrix}
\end{align*}
\begin{align*}
	h(x,u) &:= x_2
\end{align*}
%
Una volta calcolate $f$ ed $h$ si esprime~\eqref{eq:system} nella seguente forma di stato
%
\begin{subequations}\label{eq:our_system_state_form}
\begin{align}
	\begin{bmatrix}
		\dot{x}_1
		\\
		\dot{x}_2
	\end{bmatrix} &= \begin{bmatrix}
	r_s \bigl(1-\frac{x_1+x_2}{K}\bigr)\, x_1\, -m_s\, u\, x_1\, -\beta\, x_1\, +\gamma\, x_2\, -\alpha\, u\, x_1\ 
	\\
1 − ns,e+nr,e
K
 ns,e 
	r_r \bigl(1-\frac{x_1+x_2 }{K}\bigr)\, x_2\, -m_r\, u\, x_2\, +\beta\, x_1\, -\gamma\, x_2\, +\alpha\, u\, x_1\ 
\end{bmatrix} \label{eq:state_form_1}
\end{align}
\begin{align}
	y &= x_2
\end{align}
\end{subequations}
%
Per trovare la coppia di equilibrio $(x_e, u_e)$ di \eqref{eq:our_system_state_form}x si risolve il seguente sistema di equazioni
%
\begin{align}
	\begin{cases}
		r_s \bigl(1-\frac{n_{s,e}+n_{r,e}}{K}\bigr)\, n_{s,e}\, -m_s\, u_e\, n_{s,e}\, -\beta\, n_{s,e}\, +\gamma\, n_{r,e}\, -\alpha\, u_e\, n_{s,e}\, =0
		\\
		r_r \bigl(1-\frac{n_{s,e}+n_{r,e}}{K}\bigr)\, n_{r,e}\, -m_r\, u_e\, n_{r,e}\, +\beta\, n_{s,e}\, -\gamma\, n_{r,e}\, +\alpha\, u_e\, n_{s,e}\, =0
	\end{cases}
\end{align}
%
dal quale, isolando la $u(t)$, si ottiene:
%
\begin{subequations}
\begin{align}
	u_1(t)=\frac{r_s \bigl(1-\frac{n_{s,e}+n_{r,e}}{K}\bigr)\, n_{s,e}\,-\beta n_{s,e}\,+\gamma n_{r,e}}{(m_s+\alpha)\,n_{s,e}} 
	\\
	u_2(t)=\frac{r_r \bigl(1-\frac{n_{s,e}+n_{r,e}}{K}\bigr)\, n_{r,e}\,+\beta n_{s,e}\,-\gamma n_{r,e}}{m_r\,n_{r,e}+\alpha\,n_{s,e}}
\end{align}
\end{subequations}
%
calcolando quindi rispetto ai parametri forniti si ottiene
%
\begin{align}
	x_e :=\begin{bmatrix}
		n_{s,e}
		\\
		n_{r,e}
	\end{bmatrix} = 
	\begin{bmatrix}
		100
		\\
		400
	\end{bmatrix},  \quad u_e = 0 \
	\label{eq:equilibirum_pair}
\end{align}
%
Si definiscono le variabili alle variazioni $\delta x$, $\delta u$ e $\delta y$ come 
%
\begin{align*}
	\delta x \approx x(t)-x_e, \\
	\quad 
	\delta u = u(t)-u_e, \\
	\quad
	\delta y \approx y(t)-y_e. 
\end{align*}
%
L’evoluzione del sistema, espressa in termini delle variazioni delle variabili, può essere approssimativamente rappresentata attraverso il seguente sistema lineare.
%
\begin{subequations}\label{eq:linearized_system}
\begin{align}
	\delta \dot{x} &= A\delta x + B\delta u
	\\
	\delta y &= C\delta x + D\delta u,
\end{align}
\end{subequations}
%
con opportune matrici $A$, $B$, $C$ e $D$  calcolate nel seguente modo:
%
\begin{subequations}\label{eq:matrices}
\begin{align}
	A &= \begin{bmatrix}
		\frac{\partial}{\partial\,x_1}\,f_1(x,u)\bigg|_{x_e,u_e} & \frac{\partial}{\partial\,x_2}\,f_1(x,u)\bigg|_{x_e,u_e}
		\\
		\frac{\partial}{\partial\,x_1}\,f_2(x,u)\bigg|_{x_e,u_e} & \frac{\partial}{\partial\,x_2}\,f_2(x,u)\bigg|_{x_e,u_e}
	\end{bmatrix} 
	= 
	 \\ &= \begin{bmatrix}
r_s \left(1 - \frac{2x_1 + x_2}{K} \right) - m_s u_e - \beta - \alpha u_e  & \frac{r_s x_1}{K} + \gamma
		\\
\frac{r_r x_2}{K} + \beta + \alpha u_e & r_r \left( 1 - \frac{x_1+2x_2}{K} \right) - m_r u_e - \gamma
\end{bmatrix}
	=
	\begin{bmatrix}
		-1.14 & 0.54
		\\
		1.92 & -1.32
	\end{bmatrix}
	\\ \\ 
	B &= \begin{bmatrix}
		\frac{\partial}{\partial\,u}\,f_1(x,u)\bigg|_{x_e,u_e}
		\\
		\frac{\partial}{\partial\,u}\,f_2(x,u)\bigg|_{x_e,u_e}
	\end{bmatrix}
	=
	\begin{bmatrix}
		-m_s\,x_1-\alpha\,x_1
		\\
		-m_r\,x_2+\alpha\,x_1
	\end{bmatrix}
	=
	\begin{bmatrix}
		-145
		\\
		30
	\end{bmatrix}
	\\ \\
	C &= \begin{bmatrix}
		\frac{\partial}{\partial\,x_1}\,h(x,u)\bigg|_{x_e,u_e} & \frac{\partial}{\partial\,x_2}\,h(x,u)\bigg|_{x_e,u_e}
	\end{bmatrix}
	=\begin{bmatrix}
		0 & 1
	\end{bmatrix}
	\\ \\
	D &= \begin{bmatrix}
		\frac{\partial}{\partial\,u}\,h(x,u)\bigg|_{x_e,u_e}
	\end{bmatrix}
	= \begin{bmatrix}
	0
\end{bmatrix}
\end{align}
\end{subequations}
%
\begin{figure}[h!]
    \centering
    \includegraphics[width=0.7\linewidth]{Figura2.png}
    \caption{Schema di controllo~\eqref{figura2}}
    \label{figura2}
\end{figure}

\clearpage
\section{Calcolo Funzione di Trasferimento}

In questa sezione, si calcola la funzione di trasferimento $G(s)$ dall'ingresso $\delta u$ all'uscita $\delta y$, tale che $\delta Y(s) = G(s)\delta U(s)$,   mediante la seguente formula 
%
%
\begin{align}\label{eq:transfer_function}
G(s) = C(sI-A)^{-1}B-D = -521.7869\ \frac{-0.1229s+1}{(0.444s+1)(4.812s+1)}.
\end{align}
%
Dunque il sistema linearizzato~\eqref{eq:linearized_system} è caratterizzato dalla funzione di trasferimento~\eqref{eq:transfer_function} con 2 poli \\ $p_1 = -0.2078,  p_2=-2.2522$ e uno zero $z = 8.14$. In Figura \eqref{bodediagram} si mostra il corrispondente diagramma di Bode. 

\begin{figure}[h!]
    \centering
    \includegraphics[scale=0.85]{diagramma_bode.eps}
    \caption{Diagramma di Bode della funzione di trasferimento G(s)}
    \label{bodediagram}
\end{figure}

\clearpage
\section{Mappatura specifiche del regolatore}
\label{sec:specifications}

Le specifiche da soddisfare sono:
\begin{itemize}
	\item[1)] Errore a regime nullo con riferimento a gradino $w(t)=-110\cdot1(t)$ e disturbo sull'uscita $d(t)=-20\cdot1(t)$.\label{spec1}
	\item[2)] Per garantire una certa robustezza del sistema si deve avere un margine di fase $M_f\ge45^{\circ}$.\label{spec2}
	\item[3)] La sovraelongazione percentuale massima $S\%\le10\%$\label{spec3}
	\item[4)] Il tempo di assestamento al $\epsilon \% = 5\%$ deve essere inferiore al valore fissato: $T_{\alpha,\epsilon}=0.2s$.\label{spec4}
	\item[5)] Il disturbo in uscita $d(t)$ risulti attenuato di $45dB$ sapendo che la sua banda si limita a pulsazioni nel range $[0;0.1]$\label{spec5}
	\item[6)] Il rumore di misura $n(t)$ con una banda limitata nel range di pulsazioni $[5\cdot10^3;5\cdot10^6]$ deve essera abbattuto di almeno $80dB$ \label{spec6}
\end{itemize}
%
Si procede la mappatura punto per punto delle specifiche richieste. 
\begin{itemize}
	\item[1)] Per azzerare l'errore a regime in risposta ad un segnale a gradino si inserirà un polo nell'origine durante la sintesi del regolatore statico.\
	\item[2)] In corrispondenza della pulsazione critica $\omega_c$ si deve avere una fase maggiore di $-135^{\circ}$\
	\item[3)] Per ottenere una sovraelongazione percentuale massima $S\%\ge10\%$ poniamo $\xi=\frac{M_f}{100}$.
	\\
	Sapendo che la sovraelongazione massima e la $\xi$ massima rispettano la seguente relazione:\\
	 $S^\star=e^\frac{-\pi\xi^\star}{\sqrt{1-(\xi^\star)^2}}$. e quindi: \\
     \[
        \xi^* = \left| \frac{\ln{\left(\frac{S^*}{100}\right)}}{\sqrt{\pi^2 + \ln^2{\left(\frac{S^*}{100}\right)}}} \right|
        \]
        
        \[
        M_{f,S^*} = 100 \cdot \xi^*
        \]
	 \\
	 Sostituendo $\xi$ otteniamo $M_f\ge59,1^{\circ}$.
	\item[4)] Per ottenere un tempo di assestamento al $5\%$ inferiore al valore fissato  $T_{a,\epsilon}=0.2s$ è necessario imporre $e^{-\xi\omega_c\,T_a}=0.05$.\\
	Si ottiene quindi $\xi\omega_c\ge\frac{3}{T^\star}$ con $T^\star$ tempo di assestamento massimo.\\
	Da ciò si ricava quindi $\omega_c=\frac{300}{T^\star\,M_f}=25.374\frac{rad}{s}$.
	\item[5)] Per riuscire ad attenuare il disturbo in uscita $d(t)$ di 45dB risulta necessario che nell'intervallo d'interesse($[0;0.1]$) $|L(j\omega)|_{dB}\ge45dB$.\
	\item[6)] Per riuscire ad attenuare il disturbo di misura $n(t)$ di 80dB risulta necessario che nell'intervallo d'interesse($[5\cdot10^3;5\cdot10^6]$) $|L(j\omega)|_{dB}\le-80dB$.\
\end{itemize}
\clearpage
Pertanto, in Figura \ref{bodepatch}, si mostra il diagramma di Bode della funzione di trasferimento $G(s)$ con le zone proibite emerse dalla mappatura delle specifiche.

\begin{figure}[h!]
    \centering
    \includegraphics[width=1\linewidth]{bode_patch.png}
    \caption{Diagramma di Bode di G(s) con patch delle zone proibite}
	\label{bodepatch}
\end{figure}

\clearpage
\section{Sintesi del regolatore statico}
\label{sec:static_regulator}


In questa sezione si progetta il regolatore statico $R_s(s)$ partendo dalle analisi fatte in sezione~\ref{sec:specifications}.\\
La progettazione del regolatore statico è volta alla risoluzione delle problematiche dovute alle specifiche $1$ e $5$ viste nel punto precedente.\\
Per ottenere errore a regime nullo in risposta al riferimento a gradino inseriamo un polo nell'origine. Si considera inoltre un guadagno statico del nostro regolatore $\mu_s=1.5 \cdot 10^4$ per ottenere l'attenuazione richiesta di $45dB$ sul disturbo in uscita $d(t)$ a basse pulsazioni.
\\
\\
Il regolatore statico così sintetizzato risulta essere nella forma $R_s(s)=\frac{1.5 \cdot 10^4}{s}$.\\

Dunque, si definisce la funzione estesa $G_e(s) = R_s(s)G(s)$ e, in Figura \ref{geconpatch}, mostrando il suo diagramma di Bode per verificare se e quali zone proibite vengono attraversate.

\begin{figure}[h!]
    \centering
    \includegraphics[width=0.6\linewidth]{GeConPatch.png}
    \caption{Diagramma di Bode del sistema esteso con patch delle zone proibite}
    \label{geconpatch}
\end{figure}


Da Figura \ref{geconpatch} si può osservare che le specifiche riguardanti disturbo di misura $n(t)$ risultano essere rispettate dato che $|L(j\omega)|_{dB}\le-80dB$. \\
La fase del sistema esteso, non supera i $-180^{\circ}$.\\
Ciò rispetta il teorema di Bode, il cui enunciato dice che, rispettate le ipotesi riguardo l'attraversamento dello 0 in un solo punto e la mancanza di poli a parte reale positiva (ipotesi rispettate dal nostro sistema), condizione necessaria e sufficiente per la stabilità di un sistema retroazionato è che la sua $L(j\omega)$ (in questo caso la nostra $G_e(s)$) risulti avere fase maggiore di $-180^{\circ}$ alla pulsazione critica $\omega_c$.
Dalla figura si nota che non è rispettata la specifica sul margine di fase
ed è quindi necessario risollevare la fase, operazione che verrà effettuata nella sintesi del regolatore dinamico. 
